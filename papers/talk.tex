\documentstyle{slides} % -*- Dictionary: pap:papers -*- 

%\nologo	% Remove the '%' at the beginning to eliminate the logo

\begin{document}
\small
\pagestyle{headings}

\begin{titlepage}
\begin{center}
\vspace*{2.5 cm}
{\bf Gwydion Architecture}\\
{\it and}\\
{\bf Compiler Technology}\\

\vspace{2 em}
{\bf Robert A. MacLachlan}\\
\vspace{2 em}
\today
\end{center}
\end{titlepage}


\newpage
Overview

\begin{itemize}
\item System architecture
\begin{itemize}
\item Process structure
\item Hypercode and HAPI
\item Fast Development Cycle
\item Dynamism/efficiency tradeoffs
\item Compiler/Environment Integration
\end{itemize}

\item Analysis and compilation
\begin{itemize}
\item Why is optimization important?

\item Abstraction and Dynamism

\item Optimization and Tuning: accountable optimization, partial evaluation,
profiling and method selection.

\end{itemize}
\end{itemize}


\newpage
Hypercode Requirements

\begin{itemize}
\item  All previous states are included. (the design record)

\item Hypercode has semantically derived links that depend on non-local
properties of a particular project-version.  For example, which function is
called depends on which version of the defining library has been configured.

\item Hypercode requires that relationships between objects meaningfully
persist across changes to those objects.

\item In addition to slicing hypercode according to relationships present at
particular times, hypercode tools need to be able to naturally slice through
time to develop historical insight.  Queries against past semantic information
are very useful for re-engineering.
\end{itemize}


\newpage
HAPI

\begin{itemize}
\item HAPI is the Hypercode Application Programmer's Interface.  HAPI is an
extensible framework.

\item Versioning, development branch/merge and sharing control are built into
HAPI.

\item Management policy decisions are delegated to higher-level tools.

\item HAPI supports incremental update of persistent semantic information (such
as dependencies.)  HAPI triggers inference components (such as the
analyzer) when necessary.

\item This semantic data is both indexed and versioned, allowing semantic
slicing through time.
\end{itemize}


\newpage
Fast Development Cycle

\begin{itemize}
\item Studies of the advantages of dynamic prototyping environments often
find the fast design---implement---evaluate cycle to be the primary cause of
improved productivity.

\item Because of its quantifiable and immediate benefits, environments for
C++, etc., are starting to adopt incremental compilation.

\item Although Gwydion emphasizes support for new development paradigms such
as evolution and high-level programming, we still believe responsive
development tools are important.

\item Gwydion advances incremental development into new areas: concurrent
incremental design, coding, testing and documentation.  Quick transition
between all these functions is required to avoid short-term memory loss.
\end{itemize}


\newpage
Dynamism/Efficiency Tradeoffs

\begin{itemize}
\item Dynamism speeds and simplifies incremental compilation because it loosen
linkages between components.

\item If all static information is completely exploited by the compiler, the
program will be very efficient, but incremental compilation is difficult.

\item Gwydion allows the programmer to control how much static analysis
information is exploited, so the tradeoff is made depending on the stability of
each program component.
\end{itemize}


\newpage
Compiler/Environment Integration

\begin{itemize}
\item The Gwydion compiler is highly integrated with the environment.  Rather
than batch file I/O, the compiler uses hypercode and interactive tools.

\item Semantic analysis is separated from compilation.  The analysis is both
continuously updated and saved for future reference.

\item Gwydion moves beyond traditional separate compilation.  The program is
analyzed as a single unit, but analysis and compilation are done incrementally.
\end{itemize}


\newpage
Why is Optimization Important?

\begin{itemize}
\item Standard compiler optimizations have limited use because OO programs are
split into many small procedures.

\item In effect, OOP creates new opportunities for optimization (by creating
new inefficiencies.)
\end{itemize}


\newpage
Abstraction and Dynamism

OO programming depends on:

\vspace{1em}
{\bf Abstraction}
\begin{itemize}
\item Interface--Implementation separation, multiple implementations.
\item Extensibility, polymorphism and subclassing allow orderly evolution.
\item Interfaces become more general, facilitating reuse.
\end{itemize}
{\bf Dynamism}
\begin{itemize}
\item Control and data flow decisions delayed to runtime.
\item Dynamic decisions depend on runtime data instead of explicit
declarations.
\item Dynamism reduces the need for the programmer to understand certain
aspects of system behavior.
\end{itemize}


\newpage
Problems with Abstraction

Dynamism and abstraction both reduce efficiency and understandability:
\begin{itemize}
\item Dynamic decisions require runtime effort and may depend on non-obvious
properties of the entire program.

\item Abstraction barriers tend to inhibit optimization, and layers of
irrelevant abstraction make it hard to see what a method does (object
spaghetti.)  \end{itemize}


\newpage
The Gwydion/Dylan Approach

Dylan is both more dynamic and more abstract than most OO languages, which both
helps and hurts.  

\vspace{1em}
Gwydion and Dylan try to get the good without the bad:
\begin{itemize}
\item Everything is conceptually dynamic, maximizing flexibility, but
\item Static mechanisms such as type declarations and sealing are also
supported.

\item Whole-program analysis helps both compilation efficiency and user
comprehension.

\item The compiler exploits static type and sealing declarations.

\item Inefficiency is reduced by resolving abstraction and dynamism at compile
time.
\end{itemize}


\newpage
Optimization and Tuning

In Gwydion, automatic optimization and manual tuning are integrated:
\begin{itemize}
\item Eliminating unnecessary dynamism requires programmer input, but
\item The programmer doesn't understand the compiler or the complete dynamic
behavior.
\item Tools are needed to find and resolve efficiency problems.  The compiler
and programmer have a persistent dialog.
\end{itemize}


\newpage
Accountable Optimization

Accountable optimization eliminates the need for a mental efficiency model by
telling the programmer:
\begin{itemize}
\item when an inefficient implementation is chosen for non-obvious reasons,

\item how serious the situation is,

\item precisely what part of the program is being referred to, and

\item why the inefficient choice was made (and how to enable efficient
compilation.)
\end{itemize}


\newpage
Partial Evaluation

Partial evaluation is a generalization of constant folding.  It transforms
complex statements using information about parts of the program that are
constant.

\vspace{1em}
Partial evaluation comes into its own when programs are {\it written to be
optimized:}
\begin{itemize}
\item Tuned programs written in low-level languages show little benefit from
partial evaluation because they are already hand-optimized.

\item Partial evaluation helps the efficiency of abstract programs by
specializing abstract operations according to their context.

\item Partial evaluation avoids the inefficiency of dynamism by discovering
static properties.
\end{itemize}


\newpage
Dynamic Profiling

\begin{itemize}
\item During development, a dynamic profile is built of each application.

\item Profiling aids manual tuning and focuses compiler feedback.

\item Even without user input, profiling can be used to guide optimization.

\item The dynamic profile is maintained across program modifications.

\item The profile history increases representativeness and allows detection of
performance bugs.
\end{itemize}


\newpage
Method Selection

Compile-time method selection is an important form of partial evaluation, since
static method calls:
\begin{itemize}
\item need no dynamic method lookup.

\item are candidates for inline expansion, enabling more partial evaluation.

\item allow argument and result type inference.

\item can be open-coded if the method is a built-in primitive (+, etc.)
\end{itemize}

Method selection is profile-driven:
\begin{itemize}

\item Even when a method can't proven to be most applicable, profile
information can be used to guess which method will be called.

\item The most-likely method can even be inlined as long as the inline code is
guarded.
\end{itemize}


\newpage
Summary

\begin{itemize}
\item Dylan's dynamism and abstraction have the potential to significantly help
evolution and reuse, but a sophisticated environment is needed to get the most
from Dylan.

\item Hypercode and incremental semantic analysis provide support both for new
program-understanding tools and for new optimization techniques.  These tools
and optimizations minimize the negative effects of greater abstraction and
dynamism.

\item Substantial efficiency benefits can be gained by allowing the compiler
and profiler to interact with the programmer and with each other.
\end{itemize}

\end{document}
